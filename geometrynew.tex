
\documentclass[12pt]{article}
\usepackage{graphicx} % Required for inserting images
\usepackage{amsmath,amssymb,amsfonts}
\usepackage{enumitem}
\usepackage{tfrupee}
\usepackage{caption}
\usepackage{float}

\title{Geometry}
\begin{document}
\maketitle
\captionsetup[figure]{labelsep=space}
\begin{enumerate}

 \item A solid spherical ball fits exactly inside the cubical box of side 2$a$. The volume of the ball is
     \begin{enumerate}
         \item $\frac{16}{3}{\pi a^3}$
         \item $\frac{1}{6}{\pi a^3}$
         \item $\frac{32}{3}{\pi a^3}$
         \item $\frac{4}{3}{\pi a^3}$
     \end{enumerate}
  \item A frustum of a right circular cone which is of height 8 cm with radii of its circular ends as 10 cm and 4 cm, has its slant height equal to
       \begin{enumerate}
           \item 14 cm
           \item 28 cm
           \item 10 cm
           \item $\sqrt{260}$ cm
        \end{enumerate}
 \item The capacity of a cylindrical glass tumbler is 125.6 $cm^3$. If the radius of the glass tumbler is 2 cm,then finds its height.(Use $\pi=3.14$)
 
 \item A mint moulds four types of copper coins A,B,C and D whose diameters vary from 0.5 cm to 5 cm.The first coin A has a diameter of 0.7 cm.The second coin B has double the diameter of coin A and from then onwards the diameters increase by 50\%. Thickness of each coin is 0.25 cm.
     \begin{figure}[H]
         \centering
         \includegraphics[width=\columnwidth]{/storage/emulated/0/Pictures/SAVE_20231007_243006.jpg}
         \caption{}
         \label{SAVE_20231007_243006}
     \end{figure}
         After reading the above, answer the following questions:
    \begin{enumerate}
     \item  Fill in the diameters of the coins required in the following table:
    \begin{table}[H]
    \centering
    \caption{}
    \label{tab:a}
    \begin{tabular}{|c|c|}
        \hline
        Type of coin & Diameter (in cm) \\
        \hline
         A & 0.7 \\ \hline
         B & --- \\
        \hline
    \end{tabular}
\end{table}
 \item  Complete the following table:

    \begin{table}[H]
    \centering
    \caption{}
    \label{tab:b}
    \begin{tabular}{|c|c|c|}
        \hline
        Type of Coin & Area(in $cm^2$) of one face & Volume (in $cm^3$ ) \\ \hline
         
         A & 0.385 & 0.09625 \\ \hline
         B & --- & --- \\  \hline
     \end{tabular}
\end{table}
    [Use$ \pi=\frac{22}{7}$]
       
    \end{enumerate}
   
    \item A solid metallic sphere of radius 10.5 cm is melted and recast into a number of smaller cones,each of radius 3.5 cm and height 3 cm. Find the number of cones so formed.
    \item \begin{enumerate}
    \item{ In Figure \ref{SAVE_20231007_242937}, from a solid cube of side 7 cm, a cylinder of radius 2.1 cm and height 7 cm is scooped out. Find the total surface area of the remaining solid.
       \begin{figure}[H]
           \centering
            \includegraphics[width=\columnwidth]{/storage/emulated/0/Pictures/SAVE_20231007_242937.jpg}
            \caption{}
            \label{SAVE_20231007_242937}
       \end{figure}}
                              $$OR$$
   \item A well of diameter 5 m is dug 24 m deep.The earth taken out of it has been spread evenly all around it in the shape of a circular ring of width 3 m to form an embankment. Find the height of the embankment.
    \end{enumerate}
   
    \item A solid piece of metal in the form of a cuboid of dimensions 11 cm x 7 cm x 7 cm is melted to form n number of solid spheres of radii $\frac{7}{2}$ cm each. Find the value of n.
         
      \item  A 'circus' is a company of performers who put on shows of acrobats,clowns etc. to entertain people started around 250 years back,in open fields, now generally performed in tents.  \\
     one such 'Circus Tent' is shown below.
     \begin{figure}[H]
         \centering
          \includegraphics[width=\columnwidth]{/storage/emulated/0/Pictures/SAVE_20231007_243129.jpg}
         \caption{}
         \label{SAVE_20231007_243129}
     \end{figure}
         
     
     The tent is in the shape of a cylinder surmounted by a conical top. If the height and diameter of cylindrical part are 9 m and 30 m respectively and height of conical part is 8 m with same diameter as that of the cylindrical part, then find
     \begin{enumerate}
         \item the area of the canvas used in making the tent;
         \item the cost of the canvas bought for the tent at the rate \rupee 200 per sq m, if 30 sq m canvas was wasted during stitching.
     \end{enumerate}

     \item
       \begin{enumerate}
           \item 150 spherical marbles,each of diameter 1.4 cm,are dropped in a cylindrical vessel of diameter 7 cm containing some water, and are completely immersed in water. Find the rise in the level of water in the cylindrical vessel.
           \item Three cubes of side 6 cm each, are joined as shown in Figure \ref{SAVE_20231007_242943}.
           Find the total surface area of the resulting cuboid.
       \end{enumerate}
       
            \begin{figure}[H]
                \centering
                \includegraphics[width=\columnwidth]{/storage/emulated/0/Pictures/SAVE_20231007_242943.jpg}
                \caption{}
                \label{SAVE_20231007_242943}
            \end{figure}
             \item In the picture given below, one can see a rectangular in-ground swimming pool installed by a family in their backyard. There is a concrete sidewalk around the pool of width x m. The outside edges of the sidewalk measure 7 m and 12 m. The area of the pool is 36 sq.m.
            \begin{figure}[H]
                \centering
                \includegraphics[width=\columnwidth]{/storage/emulated/0/Pictures/SAVE_20231007_242949.jpg}
                \caption{}
                \label{SAVE_20231007_242949}
            \end{figure}
            \begin{enumerate}
                \item Based on the information given above, Form a quadratic equation in terms of x.
                \item Find the width of the sidewalk around the pool.
            \end{enumerate}
           
            \item John planned a birthday party for his younger sister with his friends. They decided to make some birthday caps by themselves and to buy a cake from a bakery shop. For these two item,they decided the following dimension:
           
            Cake: Cylindrical shape with diameter 24 cm and height 14 cm.\\
            Cap: Conical shape with base circumference 44 cm and height 24 cm.\\
           \begin{figure}[H]
                \centering
                \includegraphics[width=\columnwidth]{/storage/emulated/0/Pictures/SAVE_20231007_242954.jpg}
                \caption{}
                \label{SAVE_20231007_242954}
            \end{figure}
             Based on the above information, answer the following question:
       \begin{enumerate}
          \item How many square cm paper would be used to make 4 such caps ?
          \item  The bakery shop sells cake by weight (0.5 kg,1 kg,1.5 kg,etc).To have the required dimension, how much cake should they order.if 650 $cm^3$  equals 100 g of cake ?
            \end{enumerate}
              \item \begin{enumerate}
          \item The curved surface area of a right circular cylinder is 176 sq cm and its volume is 1232 cu. cm. What is the  height of the cylinder ?
             \begin{center}
             \textbf{OR}
             \end{center}
             \item The largest sphere is carved out of a solid cube of side 21 cm. Find the volume of the sphere.
        \end{enumerate}
       
         
        \item Khurja is a city in the Indian state of Uttar Pradesh famous for the pottery. Khurja pottery is traditional Indian pottery work which has attracted Indians as well as foreigners with a variety of tea-sets,crockery and ceramic tile works.A huge portion of the ceramics used in the country is supplied by Khurja and is also referred as 'The Ceramic Town' .\\

        One of the private schools of Bulandshahr organised an Educational Tour for class 10 students to Khurja. Students were very excited about the trip. Following  are the few pottery objects of Khurja.
       \begin{figure}[H]
            \centering
              \includegraphics[width=\columnwidth]{/storage/emulated/0/Pictures/SAVE_20231007_243001.jpg}
            \caption{}
            \label{SAVE_20231007_243001}
        \end{figure}
       
        Students found the shapes of the objects very interesting and they could easily relate them with mathematical shapes viz sphere, hemisphere, cylinder etc. Maths teacher who was accompanying the students asked following questions:
       
       \begin{enumerate}
           \item The internal radius of hemispherical bowl (filled completely with water) in I is 9 cm and radius and height of cylindrical jar in II is 1.5 cm and 4 cm respectively. If the hemispherical bowl is to be emptied in cylindrical jars, then how many cylindrical jars are required ?
           \item If the cylindrical jar full of water, a conical funnel of same height and same diameter is immersed, then how much water will flow out of the jar ?
       \end{enumerate}
   
       \item How many spherical shots each having diameter 3 cm can be made by melting a cuboidal solid of dimensions 18 cm x 22 cm x 6 cm ?
       \item Conical bottom tanks in which an inverted cone at the bottom is surmounted by a cylinder of same diameter, are very advantageous in industry, specially where getting every last drop from the tank is important.

       Vikas  designed a conical bottom tank where the height of the conical part is equal to its radius and the height of the cylindrical part is two times of its radius. The tank is closed from the top.
       \begin{enumerate}
           \item If the radius of the cylindrical part is 3 m, then find the volume of the tank.
           \item Find the ratio of the volume of the cylindrical part to the volume of the conical part.
       \end{enumerate}
       
       
\end{enumerate}
\end{document}


